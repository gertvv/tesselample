\documentclass[11pt]{article}
\usepackage{geometry} 
\geometry{letterpaper} 

\usepackage{graphicx}
\usepackage{amssymb}
\usepackage{amsmath}
\usepackage{amsthm}
\usepackage{bm}
\usepackage{color}

\newtheorem{theorem}{Theorem}
\newtheorem{lemma}[theorem]{Lemma}
\newtheorem{proposition}[theorem]{Proposition}
\newtheorem{corollary}[theorem]{Corollary}
\newtheorem{definition}[theorem]{Definition}
 
\newcommand{\red}[1]{\textcolor{red}{#1}}
\newcommand{\blue}[1]{\textcolor{blue}{#1}}

\newcommand{\set}[1]{\mathcal{#1} }
\newcommand{\dd}[2]{\frac{\mathrm{d} #1} {\mathrm{d} #2}}
\newcommand{\pdd}[2]{\frac{\partial #1} {\partial #2}}
\newcommand{\colvec}[2]{\left( \begin{array}{c} #1 \\ #2 \end{array} \right)} 

\title{Efficient sampling from uniform density $n$-polytopes}
\author{T. A. Trikalinos \and Gert van Valkenhoef}
%\date{} 

\begin{filecontents}{references.bib}
	@article{avis1992,
		year={1992},
		issn={0179-5376},
		journal={Discrete \& Computational Geometry},
		volume={8},
		number={1},
		doi={10.1007/BF02293050},
		title={A pivoting algorithm for convex hulls and vertex enumeration of arrangements and polyhedra},
		url={http://dx.doi.org/10.1007/BF02293050},
		publisher={Springer-Verlag},
		author={Avis, David and Fukuda, Komei},
		pages={295-313},
		language={English}
	}

	@article{hitandrun,
		author = {Tervonen, T. and van Valkenhoef, G. and Ba\c{s}t\"{u}rk, N. and Postmus, D.},
		title = {Hit-And-Run enables efficient weight generation for simulation-based multiple criteria decision analysis},
		journal = {European Journal of Operational Research},
		doi = {10.1016/j.ejor.2012.08.026},
	  year = 	 {2013},
	  volume = 	 {224},
	  number = 	 {3},
	  pages = 	 {552--559}
	}
\end{filecontents}


\begin{document}
   
\maketitle
\tableofcontents

\section{Introduction}

We are concerned with the problem of sampling from the uniform density over a convex polytope, i.e. a bounded volume in $n$-dimensional space defined by a set of linear inequality constraints.
This sampling problem has many applications, but we are primarily interested in application to sampling of weight vectors for MCDA.
Previous work has shown that the Hit-and-Run algorithm can be applied in this use case \cite{hitandrun}.
A disadvantage of Hit-and-Run is that it is a Markov Chain Monte Carlo (MCMC) algorithm, and thus there is a need for convergence checking, or oversampling to ensure convergence has been reached.
Here we explore a simple exact sampling algorithm based on a triangulation (tesselation) of the polytope.

\section{Methods}

\subsection{Definitions and requisites}

\begin{definition}[Polytope]
    A bounded convex $n$-polytope or polytope is the set of points
    \begin{equation}\label{eq:polytope}
        \set{P} = \{ \ 
            \bm{p}: \bm{A}\bm{p} \leq \bm{b} \ 
        \}
    \end{equation}
    in $\mathbb{R}^n$, 
    where $\bm{A}$ is a $r \times n$ real matrix of coefficients, $\bm{b}$ is a $r$-vector, and the relation~$\leq$ is meant elementwise.
%
%   A polytope can be determined by its vertices or extreme points $\set{V}$ ($\set{V}$ represenation of polytopes). 
\end{definition}
$\\$

\begin{definition}[Simplex]
    Let  $\bm{v}_0, \dots, \bm{v}_n$ be points in general position in $\mathbb{R}^n$. The set 
    \begin{equation}\label{eq:simplex}
        \set{S} = \Big\{ \bm{p} : \bm{p}=  a_0 \bm{v}_0 + \dots a_n \bm{v}_n, a_i \geq 0, \sum_{i=0}^{n}{a_i} = 1  \ \forall \ i=0, \dots, n \Big\}
    \end{equation}
    is a $n$-simplex or simplex in $\mathbb{R}^n$. 
    $\set{S}$ is an $n$-dimensional polytope.
\end{definition}

\noindent
More compactly, write $\bm{V}_0 = (\bm{v}_1 - \bm{v}_0, \dots , \bm{v}_n- \bm{v}_0)$ and $\bm{a} = (a_1, \dots, a_n)^T$, with $(^T)$ denoting transpose. Then 
\begin{eqnarray*}
    \bm{p} =& a_0 \bm{v}_0 + a_1 \bm{v}_1 + \dots a_n \bm{v}_n  \\
     =& (1 - \sum_{i=1}^n{a_i}) \bm{v}_0 + \sum_{i=1}^n{a_i \bm{v}_i}  \\
     =& \bm{v}_0 + \sum_{i=1}^n{a_i (\bm{v}_i -  \bm{v}_0)}  \\
     =& \bm{v}_0 + \bm{V}_0 \bm{a},  
\end{eqnarray*}
and \eqref{eq:simplex} becomes 

\begin{equation}\label{eq:simplex.mat}
    \set{S} = \{ \bm{p} : \bm{p}=  \bm{v}_0 + \bm{V}_0 \bm{a} , \ \bm{a} \geq \bm{0}, \bm{a} \bm{1}^T \leq 1 \},
\end{equation}
where $\geq, \leq$ are meant elementwise. 

\noindent
The volume of $\set{S}$ is 
\begin{equation}\label{eq:content}
    \textrm{Vol}(\set{S}) = \frac{|\det(\bm{V}_0)|}{n!}, 
\end{equation}
where $|\cdot|$ means absolute value and $\det(\cdot)$ means determinant. 
Because $\bm{v}_0, \dots, \bm{v}_n$ are in general position, rank$(\bm{V}_0)=n$ and $\det(\bm{V}_0) \neq 0$. 

$\\$

\begin{lemma}{Simplicial decomposition of polytopes} \label{le:simdecomp}$\\$
    An $n$-polytope $\mathcal{P}$ can be decomposed into $n$-dimensional simplices $\mathcal{S}_k, \ k=1,\dots K$ such that 
    $\mathcal{P} = {\mathcal{S}_1} \cup \dots \cup {\mathcal{S}_K}$ 
    and, for $k \neq l$,  
    $\mathcal{S}_k \cap \mathcal{S}_l = \emptyset$ 
    or 
    $\mathcal{S}_k \cap \mathcal{S}_l = \mathcal{T}$, where $\mathcal{T}$ is a lower-dimensional simplex. 
\end{lemma}


\begin{proof} $\\$
    \red{The proof is from textbooks or papers.}
\end{proof}


\begin{corollary}\label{cor:vol}
    If $\mathcal{S}_1, \dots, \mathcal{S}_K$ is a simplicial decomposition of $\mathcal{P}$, then $\mathrm{Vol}(\mathcal{P}) = \sum_{k=1}^K{\mathrm{Vol}(\mathcal{S}_k)}$.
\end{corollary}



\subsection{Construction of a uniform density over a simplex $\set{S}$}

\begin{theorem}\label{th:g(p)}
    Let $\bm{w}=[w_i], \ i=1, \dots, n$ be a random vector with $w_i \geq 0, w_1 + \dots + w_n \leq 1$, and density $h(\bm{w})$. 
    Then the vector $\bm{p} = \bm{v}_0 + \bm{V}_0 \bm{w} \in \set{S}$ has density 
    \begin{equation}
        g(\bm{p}) =  h(\bm{w}) |\det(\bm{V}_0)|^{-1},
    \end{equation}
\end{theorem}


\begin{proof} $\\$
    The proof is immediate from the multivariate Change of Variables Theorem:  
    \begin{equation*}
        g(\bm{p}) =  h(\bm{w}) \Big| \det \Big( {\dd{\bm{w}}{\bm{p}}} \Big) \Big| = h(\bm{w}) |\det(\bm{V}_0)|^{-1},
    \end{equation*}
    where $\Big( \dd{\bm{w}}{\bm{p}} \Big)_{ij} = \pdd{w_i}{p_j} = (\bm{V}_0^{-1})_{ij}$.
\end{proof}


To construct a uniform probability density in $\set{S}$ define a random variable $\bm{w}$ with uniform density on a regular simplex ${\set{W} = \{ w_i \geq 0, \sum{w_i} \leq 1, i=0, \dots, n\} }$. Such is a $n$-dimensional Dirichlet distribution
\begin{equation}\label{eq:dirichlet}
    h(\bm{w}) = \textrm{Dirichlet}(\bm{1}) =  (n!)^{-1}.
\end{equation}
From Theorem \ref{th:g(p)} the choice \eqref{eq:dirichlet} results in a random vector $\bm{p}$ with uniform density $g(\bm{p}) = (|\det(\bm{V}_0)| n!)^{-1}$ over $\set{S}$. 

\subsection{Construction of a uniform density over a polytope $\set{P}$}

Write 
\begin{equation}\label{eq:f(w).1}
    f(\bm{p}) = \begin{cases}
        g_k(\bm{p}) P(\bm{p} \in \set{S}_k) & \textrm{, if } \bm{p} \in \set{S}_k \subseteq \set{P} \ \forall \ k =1, \dots, K \\
        0 & \textrm{, if } \bm{p} \notin \set{P} 
        \end{cases}, 
\end{equation}
where $P(\bm{p} \in \set{S}_k)$ the probability that $\bm{p}$ belongs to the $k$-th simplex of a decomposition of $\set{P}$ as per Lemma \ref{le:simdecomp}.  Choose $P(\bm{p} \in \set{S}_k) = \textrm{Vol}(\set{S}_k) / \textrm{Vol}(\set{P})$ for all $k$. 
Then for the $k$-th simplex 
\begin{eqnarray*}\label{eq:f(w).2}
        g_k(\bm{p}) P(\bm{w} \in \set{S}_k) &=& g_k(\bm{p}) \Big( \frac{\textrm{Vol}(\set{S}_k)} {\textrm{Vol}(\set{P})} \Big) \\ 
        &=& \frac{1}{|\det(\bm{V}_{0k})| n!} \Big( \frac{|\det(\bm{V}_{0k})| / n!}{\sum_{j=1}^K{|\det(\bm{V}_{0j})|} / n!} \Big) \\
        &=& \frac{1}{ n! \ \sum_{j=1}^K{|\det(\bm{V}_{0j})|}},
\end{eqnarray*}
%
and \eqref{eq:f(w).1} becomes
%
\begin{equation}\label{eq:f(w).2}
    f(\bm{p}) = 
    \begin{cases}
        \big( n! \ \sum_{j=1}^K{|\det(\bm{V}_{0j})|} \big)^{-1} 
            & \textrm{, if } \bm{p} \in \set{P} \\
        0   & \textrm{, if } \bm{p} \notin \set{P} 
    \end{cases}. 
\end{equation}
As per \eqref{eq:f(w).2}, the construction results in a uniform density distribution over the polytope $\set{P}$.

\subsection{Algorithm for sampling}

Given the results derived above, we can sample uniformly from a convex polytope $\set{P}$ as follows:

\begin{enumerate}
	\item Find the vertices $\bm{v}_0, \dots, \bm{v}_n$ of the polytope. This can be achieved using the Avis-Fukuda pivoting algorithm \cite{avis1992}.
	\item Decompose $\set{P}$ in $n$-simplices $\set{S}_1, \dots, \set{S}_K$ using, e.g., Delaunay triangulation (any triangulation satifying Lemma \ref{le:simdecomp} is appropriate). The triangulation returns the vertices $\set{V}(\set{S}_k)$ and content Vol$(\set{S}_k)$ of each simplex $\set{S}_k$. Set $\bm{q} = (q_1, \dots, q_K)$ with $q_k = \mathrm{Vol}(\set{S}_k) / \mathrm{Vol}(\set{P})$.
    \item For the $i$-th of $N$ samples: 
        \subitem -- Draw a random vector $\bm{w}_i$ form a regular $n$-simplex
        \begin{equation}
            \bm{w}_i  \sim \textrm{Dirichlet}(\bm{1}).
        \end{equation}
        %
        \subitem -- Decide which simplex is sampled from 
        \begin{equation}
			j_i \sim \textrm{Categorical}(\bm{q}).
        \end{equation}
        %
        \subitem -- Compute the point $\bm{p}_i$ 
        \begin{equation}
            \bm{p}_i = \bm{v}_{{j_i}0} + \bm{V}_{{j_i}0} \bm{w}_i 
        \end{equation}

\end{enumerate}

\subsection{Implementation}

Our implementation is in R, using the following components for each step:
\begin{enumerate}
	\item \texttt{findVertices} from the \texttt{hitandrun} package, which in turn uses the \texttt{rcdd} package.
	\item \texttt{delaunayn} from the \texttt{geometry} package.
	\item \texttt{simplex.sample} from the \texttt{hitandrun} package to sample from the degenerate Dirichlet, and \texttt{sample} from the \texttt{base} package to sample from the Categorical.
\end{enumerate}

\subsection{Complexity}

The triangulation is by far the dominant term in both time and space complexity, because the number of simplices generated may scale as $n!$.
Hence, the algorithm is not feasible in high-dimensional space.

\section{Results}

On a 1.9 GhZ Intel Core i7-3517U CPU and 4 GB of RAM, the following running times were observed for a relatively simple polytope: \\
\begin{tabular}{rrr}
	n & t (s) & K \\
	2 & 0.159 & 1 \\
	3 & 0.154 & 5 \\
	4 & 0.159 & 44 \\
	5 & 0.162 & 210 \\
	6 & 0.227 & 2 486 \\
	7 & 0.749 & 19 763 \\
	8 & 10.439 & 359 214
\end{tabular} \\
For $n=9$, the triangulation started swapping out of RAM, and therefore no valid timing results were obtained.
Anecdotally, the running time is about three minutes for $n=9$.
Note that the triangulation only becomes a dominant cost at $n=7$, and in lower dimensions the running time could be reduced by about $40\%$ through more efficient implemenation of step 3 of the algorithm.

Rejection sampling is similarly only feasible up to about $n=8$ \cite{hitandrun} (note that their $n$ is our $n + 1$).
However, our algorithm is significantly faster for $n=7$ and $n=8$, for example rejection sampling takes over 10 minutes for $n=8$.

\bibliographystyle{abbrv}
\bibliography{references}

\end{document}
